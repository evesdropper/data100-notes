\lecture[2022-07-03]{Visualization I: Visualizing Relationships between Quantitative Variables}

\subsection{Overview}
Previously, in Data 8 and 100, visualizations were simple histograms, bar charts, and scatter plots. In the real world, visualizations can be a lot more complex, but they share teh same goals as the basic graphs we previously made:
\begin{enumerate}
\item To help your understanding of data/results.
\begin{itemize}
\item Key part of EDA and useful throughout modeling.
\item Lightweight, iterative, and flexible.
\end{itemize}
\item To communicate results/conclusions to others.
\begin{itemize}
\item Highly editorial and selective; be thoughtful and careful!
\item Fine tuned to achieve a communications goal.
\item Time consuming; bridges into design/art.
\end{itemize}
\end{enumerate}

\subsection{Distributions}
Main focus is towards visualizations of distributions.
\begin{definition}[Distribution]{A distribution describes the frequency at which values of a variable occur.
\begin{itemize}
\item All values must be accounted for once and only once.
\item Total frequencies must add up to 100\%, or the number of values that have been observed.
\end{itemize}
}
\end{definition}

\begin{example}[]{Do the following charts show a distribution?

\begin{minipage}{0.3\textwidth}
\begin{center}
\includegraphics[width=\textwidth]{7-dis1.png}
\end{center}    
\end{minipage}
\begin{minipage}{0.3\textwidth}
\begin{center}
\includegraphics[width=\textwidth]{7-dis2.png}
\end{center}    
\end{minipage}
\begin{minipage}{0.4\textwidth}
\begin{center}
\includegraphics[width=\textwidth]{7-dis3.png}
\end{center}
\end{minipage}
\tcbline 
\begin{enumerate}
\item No - one can belong in multiple categories, and it doesn't add up to 1. It just shows changes in each category.
\item No - again, one can belong in multiple categories, and it doesn't add up to 1.
\item Yes - it adds up to 1, and no one can be in multiple categories (it doesn't make sense for someone to be in both middle and upper class).
\end{enumerate}
}
\end{example}

\subsubsection{Bar Plots for Distributions}
Bar plots are the most common way of displaying the distribution of a qualitative variable (e.g. the previous example - percent of adults in each income class). Lengths encode the values, while there could be subcategories denoted by colors (not necessary), and widths do not encode anything.

Here are a few examples of bar charts using the baby weights dataset from Data 8. We made a bar chart based on the ``Maternal Smoker'' series, which can be denoted as True or False.

\textbf{Data 8 Tables}

\begin{figure}[ht]
\includegraphics[width=0.65\textwidth]{7-bar1.png}\centering\caption{Data 8 built-in charts from \mintinline{text}{datascience} library. Chose horizontal bar as it doesn't look as good with vertical bars.}
\end{figure}
\begin{minted}{python}
from datascience import Table
t = Table.from_df(births["Maternal Smoker"].value_counts().reset_index())
t.barh("index", "Maternal Smoker")
\end{minted}

\textbf{Pandas Native}

\begin{figure}[ht]
\includegraphics[width=0.65\textwidth]{7-bar2.png}\centering\caption{Pandas has built-in plotting, but we don't use it officially in Data 100.}
\end{figure}
\begin{minted}{python}
births["Maternal Smoker"].value_counts().plot(kind="bar")
\end{minted}

\textbf{Matplotlib Manual}

\begin{figure}[ht]
\includegraphics[width=0.65\textwidth]{7-bar3.png}\centering\caption{Directly use \mintinline{text}{matplotlib} is an option, but we don't do this often in Data 100.}
\end{figure}
\begin{minted}{python}
ms = births["Maternal Smoker"].value_counts()
plt.bar(ms.index, ms)
\end{minted}

\textbf{Seaborn}

\begin{figure}[ht]
\includegraphics[width=0.65\textwidth]{7-bar4.png}\centering\caption{Seaborn's \mintinline{text}{countplot} function does more abstraction; instead of already selecting out the data you want to check, it does the counting for you. Preferred approach in Data 100.}
\end{figure}
\begin{minted}{python}
import seaborn as sns
sns.countplot(data=births, x="Maternal Smoker")
\end{minted}

\textbf{Plotly}

\begin{figure}[ht]
\includegraphics[width=0.65\textwidth]{7-bar5.png}\centering\caption{Not used in Data 100, but could also be useful. Plots are interactive in plotly.}
\end{figure}
\begin{minted}{python}
import plotly.express as px
px.histogram(births, x = 'Maternal Smoker', color = 'Maternal Smoker')
\end{minted}

In short, there are many ways to generate the same plot. The first 4 are \mintinline{text}{matplotlib} based (Data 8's \mintinline{text}{Table.bar}, Pandas' \mintinline{text}{Series.bar}, Matplotlib \mintinline{text}{plt.bar}, and Seaborn \mintinline{text}{sns.countplot}). The preferred appraoch is Seaborn's \mintinline{text}{countplot} - it's simpler and provides better aesthetics. The fifth method is using \mintinline{text}{plotly}'s \mintinline{text}{px.histplot}, which is even better than Seaborn - providing a cleaner API, interactive plots, and even better aesthetics. (However, it is not taught in Data 100)

\textbf{Questions about our Bar Plot}

Some observations:
\begin{itemize}
\item Colors chosen (blue/red) are arbitrary and do not convey any more inforamtion than the existence of different categories.
\item Widths of the bars encode no useful information and are given arbitrary widths to look nice.
\end{itemize}
These prompt some questions:
\begin{itemize}
\item What colors should we use?
\item How wide should the bars be?
\item Should the legend exist?
\item Should the bars and axes have dark borders?
\end{itemize}
How do we decide on these questions? To answer these questions, let's go back to the goals of data visualization: to help our understanding of results and convery said results/conclusions to others. From there, we have our answers: for goal 1, these choices are mostly irrelevant, but for goal 2, they matter.
\begin{itemize}
\item Color choice matters when we share these results with others - some colors are more associated with negative choices (e.g. red), and could convey the meaning that smoking is bad. A better idea would be to have colors on the same gradient to stay neutral.
\item A legend should exist, especially if it contains numbers, allowing casual readers to notice.
\item A few other notes: replace True/False with Smoker/Nonsmoker; specify exact count on bar chart - don't just rely on interactivity; add an annotation to cite the data source.
\end{itemize}

\subsection{Histograms}
Another type of plot is to use the histogram to show the distribution. Suppose we wanted to plot the distribution of Maternal Pregnancy Weight as a bar plot, using each weight value as a category.

\textbf{Countplot}
\begin{figure}[ht]
\includegraphics[width=0.65\textwidth]{7-hist1.png}\centering\caption{Use each observed integer as a category. What's wrong here?}
\end{figure}
\begin{minted}{python}
sns.countplot(data=births, x="Maternal Pregnancy Weight")
\end{minted}

\textbf{Histplot}
\begin{figure}[ht]
\includegraphics[width=0.65\textwidth]{7-hist2.png}\centering\caption{Using integer bins as categories.}
\end{figure}
\begin{minted}{python}
sns.histplot(data=births, x="Maternal Pregnancy Weight")
\end{minted}

\textbf{Plotly Approach}
\begin{figure}[ht]
    \includegraphics[width=0.65\textwidth]{7-hist3.png}\centering\caption{Plotly provides interactivity, allowing you to see bins and count per bin.}
\end{figure}
\begin{minted}{python}
px.histogram(births, x="Maternal Pregnancy Weight")
\end{minted}

\textbf{Rugplots}
\begin{figure}[ht]
\includegraphics[width=0.65\textwidth]{7-rug.png}\centering\caption{Adding a rugplot to the histogram allows us to see the distribution of data points within each bin. Not useful here, but can be useful in some contexts.}
\end{figure}
\begin{minted}{python}
sns.histplot(data = births, x = 'Maternal Pregnancy Weight')
sns.rugplot(data = births, x = 'Maternal Pregnancy Weight', color = "red")
\end{minted}

\subsubsection{Evaluating Histograms}
Histograms allow us to analyze the distribution of data by their shape:
\begin{itemize}
\item \textbf{Skewness/Tails}: Skewed left/right, left/right tail.
\item \textbf{Outliers:} Determined arbitrarily.
\item \textbf{Modes}
\end{itemize}

Consider this plot:
\begin{figure}[ht]
\includegraphics[width=0.65\textwidth]{7-denscurve.png}\centering\caption{Histogram with density curve (continuous histogram).}
\end{figure}

\textbf{Skewness/Tails}
\begin{itemize}
\item If a distribution has a long right tail (like this one), it is skewed right, meaning the mean is typically to the right of the median.
\begin{itemize}
\item Think of the mean as the ``balancing point'' of the density and distribution.
\end{itemize}
\item Vice versa, if the tail is on the left, then it is skewed left.
\item The distribution can also be symmetric, where both tails are roughly equal size.
\end{itemize}

\textbf{Outliers}
\begin{itemize}
\item There are some outliers in the far right.
\item But in general, what constitutes an outlier? Just the largest point or the rightmost 4 bins?
\item Will define more carefully when covering box plots.
\end{itemize}

\textbf{Modes}
\begin{definition}[Mode]{Local/global maximum in a distribution. If there is one maximum, it is unimodal, then bimodal if there are two maximums, and multimodals if there are more than two. 
}
\end{definition}
\begin{itemize}
\item The graph looks like it could be bimodal, but how can we know for sure?
\item Sometimes the bin sizes and the presence of random noise can make it difficult to find the modes.
\begin{itemize}
\item Use a continuous distribution (density curve) instead of a discrete histogram. Then it becomes apparent that there is one mode.
\end{itemize}
\end{itemize}
\begin{figure}[ht]
\centering
\begin{minipage}{0.5\textwidth}
\centering
\includegraphics[width=\textwidth]{7-dens1.png}
\end{minipage}%
\begin{minipage}{0.5\textwidth}
\centering
\includegraphics[width=\textwidth]{7-dens2.png}
\end{minipage}
\centering\caption{If we split the histogram into smokers/nonsmokers, we can see the smoker graph is slightly bimodal, while the nonsmoker one is unimodal. This difference might be worth some more analysis.}
\end{figure}

\begin{notebox}[]
instead of labeling by counts, we can instead plot the density:
\begin{center}
\includegraphics[width=0.5\textwidth]{7-dens3.png}
\end{center}
To get density from counts and bin size:
\[
\text{Density} = \frac{\text{Count} / \text{Total}}{\text{Bin Size}}
.\] 
Vice versa, to get counts from density and bin size:
\[
    \text{Count} = \text{Bin Size} \cdot \text{Density} \cdot \text{Total Entries}
.\] 
\end{notebox}

\subsection{Box/Violin Plots}
\begin{definition}[Quartile]{A quarter of the distribution for a quantitative variable.
\begin{itemize}
\item First Quartile: 25th percentile.
\item Second Quartile: 50th percentile (median).
\item Third Quartile: 75th percentile.
\end{itemize}
}
\end{definition}
\begin{definition}[Interquartile Range]{Measures spread, middle 50\% of the data.
\[
\text{IQR} = \text{Third Quartile} - \text{First Quartile}
.\]
}
\end{definition}
Box plots are used to show the characteristics of a numerical distribution. It shows the quartiles, ``Whiskers'' at $\{\text{First Quartile} - 1.5 \text{ IQR}, \text{Third Quartile} + 1.5 \text{ IQR}\}$, Outliers (arbitrarily defined as being to the extreme of the ``whiskers'').

With box plots, it becomes easier to compare multiple distributions, but we lose a lot of information. We can prevent this with a violin plot, which is similar to a box plot but also shows the smoothened density curve, giving the box meaning.

\begin{figure}[ht]
\begin{minipage}{0.5\textwidth}
\centering 
\includegraphics[width=0.5\textwidth]{7-box.png}
\end{minipage}%
\begin{minipage}{0.5\textwidth}
\centering
\includegraphics[width=0.5\textwidth]{7-vio.png}
\end{minipage}
\centering\caption{Box plot/Violin plot for the same distribution (birth weight).}
\end{figure}
\begin{minted}{python}
sns.boxplot(y="Birth Weight", data=births)
\end{minted}

\subsection{Comparing Quantitative Distributions}
How do we compare multiple distributions? One possible option is to overlay histograms and/or density curves of two distributions.
\begin{figure}[ht]
\begin{minipage}{0.33\textwidth}
\centering
\includegraphics[width=\textwidth]{7-comp1.png}
\end{minipage}%
\begin{minipage}{0.33\textwidth}
\centering
\includegraphics[width=\textwidth]{7-comp1.png}
\end{minipage}%
\begin{minipage}{0.33\textwidth}
\centering
\includegraphics[width=\textwidth]{7-comp3.png}
\end{minipage}
\centering\caption{Overlaying histograms and/or density curves.}
\end{figure}
\begin{minted}{python}
# First approach 
sns.distplot(nsm_bweights, bins=bw_bins, kde=False, hist_kws=dict(ec='w'), label='non smoker')
sns.distplot(sm_bweights, bins=bw_bins, kde=False, hist_kws=dict(ec='w'), label='smoker')
# Second approach
sns.distplot(nsm_bweights, bins=bw_bins, hist_kws=dict(ec='w'), label='non smoker')
sns.distplot(sm_bweights, bins=bw_bins, hist_kws=dict(ec='w'), label='smoker')
# Third approach
sns.distplot(nsm_bweights, hist=False, label='non smoker')
sns.distplot(sm_bweights, hist=False), label='smoker')
\end{minted}

Instead of using histograms, which do not generalize as well, we can use side-by-side box and violin plots. Since they are more concise, it is easier to compare the distributions with such plots.
\begin{itemize}
\item We can easily tell that median birth weight for non smokers was higher.
\item Using the violin plot, we can see that the smoker category (True) is bimodal.
\end{itemize}

\begin{figure}[ht]
\begin{minipage}{0.5\textwidth}
    \centering
\includegraphics[width=0.5\textwidth]{7-bcomp.png}
\end{minipage}%
\begin{minipage}{0.5\textwidth}
    \centering
\includegraphics[width=0.5\textwidth]{7-vcomp.png}
\end{minipage}
\centering
\caption{Side by side box/violin plots.}
\end{figure}
\begin{minted}{python}
sns.boxplot(data=births, x='Maternal Smoker', y='Birth Weight')
sns.violinplot(data=births, x='Maternal Smoker', y='Birth Weight')
\end{minted}

\subsection{Relationships between Quantitative Variables}
Currently, only discussed visualizations for distributions of variables, but we can also use visualizations to show if there is a relationship between two variables.

Some example relationships:
\begin{center}
\includegraphics[width=0.5\textwidth]{7-rel.png}
\end{center}
Note that the bottom left one is linear spreading as it is linear for the most part, but spreads as x increases. 

The relationship between the data allows us to decide which type of model to build - e.g. the first two we can use linear regression, while other techniques are used for nonlinear relationships.
