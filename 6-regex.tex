\lecture[2022-06-25]{Regular Expressions}

Previously, in Mauna Loa example, we worked with numeric data that had missing values. There were 3 options:
\begin{itemize}
\item Drop missing records.
\item Use NaN (not a number) for missing values.
\item Impute missing values with the interpolated \mintinline{text}{Int} column.
\end{itemize}
\begin{center}
\includegraphics[width=0.65\textwidth]{6-impute.png}
\end{center}

With numerical data, you do data wrangling with EDA, but with text data, wrangling is upfront and requires Python string manipulation and Regex.

\textbf{Why work with text?}
\begin{enumerate}
    \item \textbf{Canonicalization:} Convert data that has more than one possible presentation into a standard form (e.g. join tables with mismatched labels)
\item \textbf{Extract} information into a new feature (e.g. extract dates and times from log files).
\end{enumerate}

\subsection{Python String Methods}
A possible way we can work with text data is with Python string methods to parse/split/replace strings. Here are a few examples.
\begin{example}[Canonicalization]{How do we make all the county names standardized (lower case, no spaces, remove punctuation, etc.)?
\begin{center}
\includegraphics[width=0.5\textwidth]{6-canonic.png}
\end{center}
\tcbline 
\begin{minted}{python}
def canonicalize_county(county_name):
    return (county_name
    .lower() # lowercase
    .replace(' ', '')
    .replace('&', 'and') # remove spaces/dots, change & to and
    .replace('.', '')
    .replace('county', '')
    .replace('parish', '') # remove county/parish
    )
\end{minted}
}
\end{example}

\begin{example}[Extraction]{How do we extract specific days, etc. from a log file date?
\begin{center}
\includegraphics[width=0.5\textwidth]{6-extract.png}
\end{center}
\tcbline 
\begin{minted}{python}
pertinent = line.split("[")[1].split(']')[0] # get the date within brackets - get after the opening bracket and split off anything after closing bracket
day, month, rest = pertinent.split('/') # split by /
year, hour, minute, rest = rest.split(':') # split YYYY:HH:MM
seconds, time_zone = rest.split(' ') # split the rest
\end{minted}
}
\end{example}

\textbf{String Methods Reference}
\begin{center}
\begin{tabular}{@{}ccc@{}}
\toprule
    Operation & Python & Pandas Series \\
\midrule
    Transformation & \mintinline{text}{s.lower()/s.upper()} & \mintinline{text}{ser.str.lower()/ser.str.upper()} \\
    Replacement/Deletion & \mintinline{text}{s.replace(<find>, <repl=''>)} & \mintinline{text}{ser.str.replace(<find>, <repl=''>)} \\
    Split & \mintinline{text}{s.split(<keyword>)} & \mintinline{text}{ser.str.split(<keyword>)} \\
    Substring & \mintinline{text}{s[1:4]} & \mintinline{text}{ser.str[1:4]} \\
    Membership & \mintinline{text}{'ab' in s} & \mintinline{text}{ser.str.contains('ab')} \\
    Length & \mintinline{text}{len(s)} & \mintinline{text}{ser.str.len()} \\
\bottomrule
\end{tabular}

\end{center}

However, Python string functions are very brittle - it requires maintenance if the data changes, and it is also not as flexible (e.g. finding all the \mintinline{text}{moo+n} patterns in a string would be difficult). This often makes the extraction and canonicalization functions very ``hacky'' and messy.

An alternative is to use Regex, or regular expressions with the Python \mintinline{text}{re} library and the pandas \mintinline{text}{str} accessor.

\begin{example}[Extraction, Revisited]{How can we change the code from the log file to a regular expression?
\tcbline 
\begin{minted}{python}
import re
pattern = r'\[(\d+)\/(\w+)\/(\d+):(\d+):(\d+):(\d+) (.+)\]'
day, month, year, hour, minute, seconds, time_zone = re.findall(pattern, line)[0]
\end{minted}
}
\end{example}

\subsection{Regex}
\begin{definition}[Formal Language]{Set of strings, typically described implicitly. (e.g. set of all strings with \mintinline{text}{len(s) < 10} with the phrase \mintinline{text}{'data'})
}
\end{definition}
\begin{definition}[Regular Language]{Formal language that can be described by a regular expression.
}
\end{definition}
\begin{definition}[Regular Expression]{Sequence of characters that specifies a search pattern.

e.g. to describe SSNs, we can use the following sequence: \mintinline{text}{[0-9]\{3\}-[0-9]\{2\}[0-9]\{4\}} (3 of any digit, dash, 2 digits, dash, 4 digits)
}
\end{definition}

\textbf{Basic Regex Syntax}

The four basic operators for Regex - you can do anything with these 4 operations.
\begin{center}
\begin{tabular}{@{}ccccc@{}}
\toprule
    Operation & Order & Example & Matches & Doesn't Match \\
\midrule
    Concatenation & 3 & \mintinline{text}{AABAAB} & \mintinline{text}{AABAAB} & every other string \\
    Or & 4 & \mintinline{text}{AA|BAAB} & either \mintinline{text}{AA, BAAB} & every other string \\
    Closure (zero or more) & 2 & \mintinline{text}{AB*A} & \mintinline{text}{AA, ABBBBBBA} & \mintinline{text}{AB, ABABA} \\
    Group (parentheses) & 1 & \mintinline{text}{A(A|B)AAB, (AB)*A} & \mintinline{text}{A, ABABABABA} & \mintinline{text}{AA, ABBA} \\
\bottomrule
\end{tabular}
\end{center}
\begin{notebox}[]
Note that \mintinline{text}{|, *, ()} are metacharacters, only manipulating adjacent characters.

e.g. \mintinline{text}{AB*} means one A, then 0 or more Bs, while \mintinline{text}{(AB)*} means 0 or more sets of AB
\end{notebox}
\begin{example}[]{Find a regular expression that matches ``moon'', ``moooon'' and so on (nonzero even \# of o).
\tcbline 
\begin{minted}{python}
pattern = r"moo(oo)*n"
\end{minted}
\tcbline
Follow up: find a regular expression that matches any nonzero even number of o or u.
\begin{minted}{python}
pattern = r"m(oo|uu)(oo|uu)*n"
\end{minted}
}
\end{example}

\textbf{Expanded Regex Syntax}
\begin{center}
\begin{tabular}{@{}cccc@{}}
\toprule
    Operation & Example & Matches & Doesn't Match \\
\midrule
    Any character (except \mintinline{text}{\n}) & \mintinline{text}{.U.U.U.} & \mintinline{text}{CUMULUS, JUGULUM} & \mintinline{text}{SUCCUMBUS, TUMULTUOUS} \\
    Character class & \mintinline{text}{[A-Za-z][a-z]*} & \mintinline{text}{word, Capitalized} & \mintinline{text}{camelCase, 4illegal} \\
    Repeat \mintinline{text}{a} times & \mintinline{text}{j[aeiou]\{3\}hn} & \mintinline{text}{jaoehn, jooohn} & \mintinline{text}{jhn, jaeiouhn}  \\
    Repeat \mintinline{text}{\{a, b\}} times & \mintinline{text}{j[ou]\{1,2\}hn} & \mintinline{text}{john, juohn} & \mintinline{text}{jhn, jooohn} \\
    At least 1 & \mintinline{text}{jo+hn} & \mintinline{text}{john, jooohn} & \mintinline{text}{jhn, jjohn} \\
    Zero or One & \mintinline{text}{joh?n} & \mintinline{text}{jon, john} & Any other string \\
\bottomrule
\end{tabular}
\end{center}

\begin{example}[]{Find a regular expression for the following:
\begin{enumerate}
\item Any lowercase string with a repeated vowel (e.g. ``noon'', ``peel'', etc.).
\item String containing a lowercase letter and a number.
\end{enumerate}
\tcbline 
\begin{minted}{python}
pattern1 = r"[a-z]*(aa|ee|ii|oo|uu)[a-z]*"
pattern2 = r"(.*[a-z].*[0-9].*)|(.*[0-9].*[a-z].*)"
\end{minted}
}
\end{example}

\textbf{Convenient Regex Syntax}
\begin{center}
\begin{tabular}{@{}cccc@{}}
\toprule
    Operation & Example & Matches & Doesn't Match \\
\midrule
    Built-in char classes & \mintinline{text}{\w+, \d+, \s+} & \mintinline{text}{Fawef_03, 12312, <whitespace>} & \mintinline{text}{a cup, 423 p, <non-ws>} \\
    Char class negation (not in char class) & \mintinline{text}{[^a-z]+} & \mintinline{text}{PEPPERS191, 1711!} & \mintinline{text}{porch, Clams} \\
    Escape Character (char literally) & \mintinline{text}{cow\.com} & \mintinline{text}{cow.com} & \mintinline{text}{cowscom} \\
\bottomrule
\end{tabular}
\end{center}
Note: \mintinline{text}{\w = [A-Za-z0-9_], \d = [0-9], \s = <whitespace>}.

\begin{example}[Log File, Revisited]{Give a regular expression to match the date portion in the log file.
\tcbline
\begin{minted}{python}
pattern = r"\[.*\]"
\end{minted}
}
\end{example}

\textbf{Additional Regex Features}
\begin{center}
\begin{tabular}{@{}cccc@{}}
\toprule
    Operation & Example & Matches & Doesn't Match \\
\midrule
    Beginning of Line & \mintinline{text}{\^ark} & \mintinline{text}{ark two, ark o ark} & \mintinline{text}{dark} \\
    End of Line & \mintinline{text}{ark\$} & \mintinline{text}{dark, ark o ark} & \mintinline{text}{ark two} \\
    Lazy version of 0+/\mintinline{text}{*} & \mintinline{text}{5.*?5} & \mintinline{text}{5005, 55} & \mintinline{text}{500505} \\
\bottomrule
\end{tabular}
\end{center}

\subsubsection{Regex in Python/Pandas}

\textbf{Canonicalization}
\textbf{Python: }We can use the \mintinline{text}{re.sub} function from the \mintinline{text}{re} module, replacing all instances of \mintinline{text}{pattern} with \mintinline{text}{repl}.

Example: 
\begin{minted}{python}
text = "<div><td valign="top">Moo</td></div>"
pattern = r"<[^>]+>" # html tags
re.sub(pattern, '', text) # replaces html tags with empty char, returns Moo
\end{minted}

\textbf{Pandas:} Use \mintinline{text}{ser.str.replace(pattern, repl, regex=True)} to replace all instances of \mintinline{text}{pattern} with \mintinline{text}{repl}.

Example:
\begin{minted}{python}
pattern = r"<[^>]+>"
df["Html"].str.replace(pattern, '', regex=True)
\end{minted}

\begin{notebox}[Raw Strings]
Use raw strings \mintinline{text}{r''/r""} as opposed to regular strings when using Regex. Both Regex and Python use backslash \mintinline{text}{\\} to escape characters, so it would lead to much more cluttered and uglier regular expressions (e.g. \mintinline{text}{"\\\\section"} vs \mintinline{text}{r"\\section"})
\end{notebox}

\textbf{Extraction}

\textbf{Python:} Use \mintinline{text}{re.findall(pattern, text)} to find all instances of a pattern in the text.

Example:
\begin{minted}{python}
text = "My social security number is 123-45-6789 bro, or actually maybe it’s 321-45-6789.";
pattern = r"[0-9]{3}-[0-9]{2}-[0-9]{4}"
re.findall(pattern, text) # finds all valid numbers that could be SSNs
\end{minted}

\textbf{Pandas:} \mintinline{text}{ser.str.findall(pattern)} returns a series of lists of valid matches.

Example:
\begin{minted}{python}
pattern = r"[0-9]{3}-[0-9]{2}-[0-9]{4}"
df["SSN"].str.findall(pattern)
\end{minted}

\textbf{Capture Groups}

Parentheses were used to specify the order of operations at first, but they also thought of as a match/capture group, which are returned as tuples of groups.

Example:
\begin{minted}{python}
text = """Observations: 03:04:53 - Horse awakens.
03:05:14 - Horse goes back to sleep."""
pattern = "(\d\d):(\d\d):(\d\d) - (.*)"
matches = re.findall(pattern, text)
# returns [('03', '04', '53', 'Horse awakens.'), ('03', '05', '14', 'Horse goes back to sleep.')]
\end{minted}

\begin{notebox}[Findall vs. Extract]
Another function is \mintinline{text}{ser.str.extract(pattern)} and \mintinline{text}{ser.str.extractall(pattern)} which returns a dataframe of the first match (one group per column) / of all matches, one group per column, one row per match.
\end{notebox}

\begin{notebox}[Limitations for Regex]
Writing regex is like writing a program: you need to know the syntax well, can be easier to write than read, and can be difficult to debug.

Regex is also not the best at some problems:
\begin{itemize}
\item Parsing hierarchical structures (e.g. JSON)
\item Complex Features (e.g. valid email addresses)
\item Counting (e.g. find same numbers of two characters, which is impossible in regex)
\item Complex properties (e.g. palindromes, balanced parentheses, which are impossible in regex)
\end{itemize}

All in all, regex is still decent at wrangling text data.
\end{notebox}

\subsection{Restaurant Data Demo}


